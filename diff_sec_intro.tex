% !TEX root = paper.tex

\section{Introduction}

\DIFaddbegin \DIFadd{The theory of the strong interaction, }\DIFaddend Quantum chromodynamics (QCD) \DIFdelbegin \DIFdel{, describing the strong interaction, predicts a deconfined phase of matters, }\DIFdelend \DIFaddbegin \DIFadd{predicts a strongly interacting matter that is so-called }\DIFaddend quark-gluon plasma (QGP) in extremely high temperature and \DIFdelbegin \DIFdel{energy density}\DIFdelend \DIFaddbegin \DIFadd{energy-density conditions}\DIFaddend . Such conditions \DIFdelbegin \DIFdel{are achieved }\DIFdelend \DIFaddbegin \DIFadd{have reached a climax  }\DIFaddend with ultra-relativistic \DIFdelbegin \DIFdel{collisions of heavy ions }\DIFdelend \DIFaddbegin \DIFadd{heavy-ion collisions }\DIFaddend at the Large Hadron Collider (LHC)\DIFdelbegin \DIFdel{~\mbox{%DIFAUXCMD
\cite{Wolschin:2020qxa}}\hspace{0pt}%DIFAUXCMD
.}\DIFdelend \DIFaddbegin \DIFadd{.%DIF > ~\cite{Wolschin:2020qxa} 
}\DIFaddend Many observations at the LHC such as flow modulations~\cite{Bhalerao:2020ulk, ALICE:2019zfl}, jet quenching~\cite{ALICE:2019qyj}, and nuclear modification factors~\cite{ALICE:2019hno} support the existence of the QGP~\cite{Adams:2005dq}\DIFaddbegin \DIFadd{, }\DIFaddend so far. 
\DIFdelbegin %DIFDELCMD < 

%DIFDELCMD < %%%
\DIFdel{The nuclear modification factor is one of the strongest tool to prove the existence of QGP. The }\DIFdelend %DIF > Specifically, the nuclear modification factor is one of the strongest tool to study the properties of the QGP.
\DIFaddbegin \DIFadd{Specifically, the }\DIFaddend nuclear modification factors for different particles in heavy-ion collisions show that the behaviour of particles \DIFdelbegin \DIFdel{are strongly modified as they interact with }\DIFdelend \DIFaddbegin \DIFadd{is strongly modified in the }\DIFaddend hot and dense medium \DIFdelbegin \DIFdel{. The nuclear modification factor, on the other hand, }\DIFdelend \DIFaddbegin \DIFadd{while those }\DIFaddend in p--A collisions~\cite{ALICE:2016dei} \DIFdelbegin \DIFdel{is }\DIFdelend \DIFaddbegin \DIFadd{are }\DIFaddend measured as unity in minimum-bias events \DIFdelbegin \DIFdel{and this states that there is }\DIFdelend \DIFaddbegin \DIFadd{stating }\DIFaddend no strong modification in p--A collisions \DIFdelbegin \DIFdel{at high }\DIFdelend \DIFaddbegin \DIFadd{in a  high transverse momentum range~(}\DIFaddend $p_{\rm{T}}>$~3~GeV/$c$\DIFaddbegin \DIFadd{)}\DIFaddend . 

On the other hand, \DIFdelbegin \DIFdel{he }\DIFdelend \DIFaddbegin \DIFadd{the }\DIFaddend short-lived resonances such as $\rm{}K^{*}$(892)~\cite{ALICE:2019etb, ALICE:2016sak}, $\rho$(770)~\cite{ALICE:2018qdv}, and $\Lambda$(1520)~\cite{ALICE:2018ewo} are good probes to study the properties of the dynamic evolution of the fireball~\cite{Bierlich:2021poz}. The evolution time from \DIFdelbegin \DIFdel{chemical out to kinetic freeze-out is comparable }\DIFdelend \DIFaddbegin \DIFadd{the chemical to kinetic freeze-outs is compatible }\DIFaddend with the lifetime of resonances~\cite{ALICE:2011dyt, ALICE:2019xyr}. The daughter particles from them can actively interact with the hadronic gas, resulting in modifications of resonance yields. Such modifications can be observed by comparing the yield of resonances with \DIFdelbegin \DIFdel{the yield }\DIFdelend \DIFaddbegin \DIFadd{those }\DIFaddend of long-lived or ground-state particles~\cite{ALICE:2018pal}. Measurements of $\rho/\pi$ and $\rm{}K^{*}/K$ yield ratios are good examples to study the properties of the late hadronic phase after \DIFaddbegin \DIFadd{the }\DIFaddend chemical freeze-out.

Light scalar mesons, whose spin and parity are zero and even respectively, are of particular interest as their natures can be explained with an exotic structure~\cite{ParticleDataGroup:2020ssz}. The understanding of them lies in a long-standing puzzle, where many discussions can be found in \DIFdelbegin \DIFdel{~\mbox{%DIFAUXCMD
\cite{ExHIC:2010gcb, Jaffe:1976ig, Maiani:2004uc}}\hspace{0pt}%DIFAUXCMD
.Among the mesons, }\DIFdelend \DIFaddbegin \DIFadd{Refs.~\mbox{%DIFAUXCMD
\cite{ExHIC:2010gcb, Jaffe:1976ig, Maiani:2004uc}}\hspace{0pt}%DIFAUXCMD
. The structure of }\DIFaddend \fzero\DIFdelbegin \DIFdel{~,one of scalar mesons, is observed many years ago in $\pi\pi$ scattering experiments~\mbox{%DIFAUXCMD
\cite{Mennessier:2010xz}}\hspace{0pt}%DIFAUXCMD
. Despite a long history of experimental observation so far and theoretical research, the nature of such a short-lived resonance is not understood and there is no consensus on its internal structure, where the structure }\DIFdelend \DIFaddbegin \DIFadd{\ }\DIFaddend is suggested to be \DIFaddbegin \DIFadd{a }\DIFaddend normal $\rm{}q\bar{q}$~\cite{Chen:2003za}, compact tetraquark~\cite{Achasov:2020aun}, or $\rm{}K\overline{K}$ molecule~\cite{Ahmed:2020kmp}. \DIFdelbegin \DIFdel{Moreover, }%DIFDELCMD < \fzero%%%
\DIFdel{~production is also interesting due to the short lifetime~\mbox{%DIFAUXCMD
\cite{ParticleDataGroup:2020ssz}}\hspace{0pt}%DIFAUXCMD
, which is a good tool to study the hadronic final state and the particle formation procedure. One of models predicting the particle formation is the Canonical Statistical Model (CSM)~\mbox{%DIFAUXCMD
\cite{Vovchenko:2019kes}}\hspace{0pt}%DIFAUXCMD
, which considers system-size dependent hadrochemistry at vanishing baryon density with local conservation of electric charged, baryon density, and strangeness, while allowing for undersaturation of strangeness}\DIFdelend \DIFaddbegin \DIFadd{In order to analyse the unknown structure of }\fzero\DIFadd{, several methods can be employed}\DIFaddend . 

The \DIFdelbegin \DIFdel{number of constituent quarks (NCQ)~\mbox{%DIFAUXCMD
\cite{Fries:2003vb} }\hspace{0pt}%DIFAUXCMD
can be suggested by measuring the flow modulations and the nuclear modification factor for different particle. The elliptic flows scaled down by the NCQ are comparable between different mesons and baryons~\mbox{%DIFAUXCMD
\cite{Wang:2022det}}\hspace{0pt}%DIFAUXCMD
. Application of the NCQ scaling to }\DIFdelend \DIFaddbegin \DIFadd{relative production rates of particles containing strange quark is reported to increase faster than those with up and down quarks that is usually referred to as ``strangeness enhancement''~\mbox{%DIFAUXCMD
\cite{ALICE:2016fzo}}\hspace{0pt}%DIFAUXCMD
. The study of relative production rate for }\DIFaddend \fzero\DIFdelbegin \DIFdel{~is expected to provide insights into the nature of }\DIFdelend \DIFaddbegin \DIFadd{\ can be useful to check the strange quark content inside }\DIFaddend \fzero. \DIFdelbegin \DIFdel{The }\DIFdelend \DIFaddbegin \DIFadd{On the other hand, the }\DIFaddend nuclear modification factors for baryons \DIFdelbegin \DIFdel{exhibit }\DIFdelend \DIFaddbegin \DIFadd{exhibits a }\DIFaddend strong enhancement at intermediate $p_{\mathrm{T}}$ (2~$<p_{\mathrm{T}}<$~5~GeV/$c$) compared with one for \DIFdelbegin \DIFdel{meson }\DIFdelend \DIFaddbegin \DIFadd{mesons }\DIFaddend which is mainly due to the \DIFdelbegin \DIFdel{NCQ~\mbox{%DIFAUXCMD
\cite{Cronin:1974zm}}\hspace{0pt}%DIFAUXCMD
. Measurement of the nuclear modificaction factor for }\DIFdelend \DIFaddbegin \DIFadd{number of constituent quarks (NCQ)~\mbox{%DIFAUXCMD
\cite{Cronin:1974zm,Fries:2003vb}}\hspace{0pt}%DIFAUXCMD
. This approach can help us to constrain the number of quarks forming }\DIFaddend \fzero\DIFdelbegin \DIFdel{, therefore, can suggest the internal structure of }\DIFdelend \DIFaddbegin \DIFadd{. However, these studies need to be approached carefully as the }\DIFaddend \fzero\DIFdelbegin \DIFdel{.
}\DIFdelend \DIFaddbegin \DIFadd{\ production is also affected during the hadronic final state and particle formation procedure due to its short lifetime~\mbox{%DIFAUXCMD
\cite{ParticleDataGroup:2020ssz}}\hspace{0pt}%DIFAUXCMD
.  
}

%DIF > The number of constituent quarks (NCQ)~\cite{Fries:2003vb} are supposed to affect to the flow modulations and the nuclear modification factor for different particle. The nuclear modification factors for baryons exhibits a strong enhancement at intermediate $p_{\mathrm{T}}$ (2~$<p_{\mathrm{T}}<$~5~GeV/$c$) compared with one for mesons which is mainly due to the NCQ~\cite{Cronin:1974zm}. 

%DIF > On the other hand, \fzero~production is also interesting due to its short lifetime~\cite{ParticleDataGroup:2020ssz} that can be also useful to study the hadronic final state and particle formation procedure.


%DIF > The elliptic flows scaled down by the NCQ are known to be comparable between different mesons and baryons~\cite{Wang:2022det}. Application of the NCQ scaling to \fzero\ is expected to provide insights into the nature of \fzero. The nuclear modification factors for baryons exhibits a strong enhancement at intermediate $p_{\mathrm{T}}$ (2~$<p_{\mathrm{T}}<$~5~GeV/$c$) compared with one for mesons which is mainly due to the NCQ~\cite{Cronin:1974zm}. Measurement of the nuclear modificaction factor for \fzero, therefore, can suggest the internal structure of \fzero.

 %DIF > One of models predicting the particle formation is the Canonical Statistical Model (CSM)~\cite{Vovchenko:2019kes}, which considers a system-size dependent chemistry at vanishing baryon density with conserved electric charge, baryon density, and strangeness locally, while allowing for under-saturation of strangeness.
\DIFaddend 

In this paper, multiplicity dependent measurement of abnormal suppression of \fzero~production in p--Pb collisions at $\snn$ = 5.02~TeV is reported. The measurement of \fzero~is conducted at midrapidity (-0.5~$<\mathrm{y}<$~0) in a transverse momentum ($p_{\mathrm{T}}$) range of 0~$<p_{\mathrm{T}}<$~8~GeV/$c$ for different multiplicity classes. In Sec.~\ref{sec:setup}, ALICE apparatus is described, and the reconstruction of \fzero~and correction is explained in Sec.~\ref{sec:ana}. Systematic study for the measurement is reported in Sec.~\ref{sec:syst}. In Sec.~\ref{sec:results}, $p_{\mathrm{T}}$ spectra, particle yield ratios, and the nuclear modification factors are discussed. Finally, model comparisons and conclusions are described in Sec.~\ref{sec:summary}.

\label{sec:intro}



