% !TEX root = paper.tex

\section{Introduction}

Quantum chromodynamics (QCD), describing the strong interaction, predicts a deconfined phase of matters, quark-gluon plasma (QGP) in extremely high temperature and energy density. Such conditions are achieved with ultra-relativistic collisions of heavy-ions at the Large Hadron Collider (LHC)~\cite{Wolschin:2020qxa}. And many observations such as flow modulations~\cite{Bhalerao:2020ulk, ALICE:2019zfl}, jet quenching~\cite{ALICE:2019qyj}, and nuclear modification factors~\cite{ALICE:2019hno} support the existence of QGP~\cite{Adams:2005dq} so far.

Among many observations stating the existence of QGP is the nuclear modification factor. The nuclear modification factors for different particles in heavy-ion collisions show many particles are strongly modified as they interact with hot and dense medium. The nuclear modification factor, on the other hand, in p--A collisions~\cite{ALICE:2016dei} is measured as unity and this states that there is no strong modification in p--A collisions at high $p_{\rm{T}}>$~3~GeV/$c$.

The short-lived resonances such as $K^{*}$(892)~\cite{ALICE:2019etb, ALICE:2016sak}, $\rho$(770)~\cite{ALICE:2018qdv}, and $\Lambda$(1520)~\cite{ALICE:2018ewo} are good probes to study the properties of the dynamic evolution of the fireball~\cite{Bierlich:2021poz}. The evolution time from chemical freeze out to kinetic freeze out is comparable with the lifetime of resonances~\cite{ALICE:2011dyt, ALICE:2019xyr}. The daughter particles from them can actively interact with the hadronic gas, resulting in modifications of resonance yields. Such modifications can be observed by comparing the yield of resonances with the yield of long-lived or ground state particles~\cite{ALICE:2018pal}. Measurements of $\rho/\pi$ and $K^{*}/K$ yield ratios are good examples to study the properties of the late hadronic phase after chemical freeze out.

Light scalar mesons, whose spin, parity, and charge are zero, even, and $J^{PC} = 0^{++}$, are of particular interesting as their natures can be explained with an exotic structure~\cite{ParticleDataGroup:2020ssz}. The understanding of them lies in a long-standing puzzle, where many discussions can be found in~\cite{ExHIC:2010gcb, Jaffe:1976ig, Maiani:2004uc}. Among the mesons, \fzero~,one of scalar mesons, is observed many years ago in $\pi\pi$ scattering experiments~\cite{Mennessier:2010xz}. Despite a long history of experimental observation and theoretical research, the nature of such a short-lived resonance is not understood and there is no consensus on its internal structure, where the structure is suggested to be normal $q\bar{q}$~\cite{Chen:2003za}, compact tetraquark~\cite{Achasov:2020aun}, or $K\overline{K}$ molecule~\cite{Ahmed:2020kmp}. Moreover, \fzero~production is also interesting due to the short lifetime~\cite{ParticleDataGroup:2020ssz}, which is a good tool to study the hadronic final state and the particle formation procedure. One of models predicting the particle formation is the Canonical Statistical Model (CSM)~\cite{Vovchenko:2019kes}, which considers system-size dependent hadrochemistry at vanishing baryon density with local conservation of electric charged, baryon density, and strangeness, while allowing for undersaturation of strangeness.

The number of constituent quarks (NCQ)~\cite{Fries:2003vb} can be suggested by measuring the flow modulations and the nuclear modification factor for different particle. The elliptic flows scaled down by the NCQ are comparable between different mesons and baryons~\cite{Wang:2022det}. Application of the NCQ scaling to \fzero~is expected to provide insights into the nature of \fzero. The nuclear modification factors for baryons exhibit strong enhancement at intermediate $p_{\mathrm{T}}$ (2~$<p_{\mathrm{T}}<$~5~GeV/$c$) compared with one for meson which is mainly due to the NCQ~\cite{Cronin:1974zm}. Measurement of the nuclear modificaction factor for \fzero, therefore, can suggest the internal structure of \fzero.

In this paper, multiplicity dependent observation of abnormal suppression of \fzero~production in p--Pb collisions at $\snn$ = 5.02~TeV is reported. Measurement of \fzero~is conducted at midrapidity (-0.5~$<\mathrm{y}<$~0) for the 0~$<p_{\mathrm{T}}<$~8~GeV/$c$ range in different multiplicity classes. In Sec.~\ref{sec:setup}, ALICE apparatus is described, and the reconstruction of \fzero~and correction is explained in Sec.~\ref{sec:ana}. Systematic study for the measurement and corresponding systematic uncertainty is reported in Sec.~\ref{sec:syst}. In Sec.~\ref{sec:results}, $p_{\mathrm{T}}$ spectrum, particle yield ratios, and the nuclear modification factors are discussed and compared with model descriptions, and conclusions is given in Sec.~\ref{sec:summary}.

\label{sec:intro}



