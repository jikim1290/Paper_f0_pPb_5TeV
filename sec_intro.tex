% !TEX root = paper.tex

\section{Introduction}

Light scalar mesons, whose spin and parity are zero and even, respectively, are of particular interest as their nature can be explained with an exotic structure~\cite{ParticleDataGroup:2022pth}. Among them, the understanding of \fzero\ particles lies in a long-standing puzzle in terms of its quark content~\cite{ExHIC:2010gcb, Jaffe:1976ig, Maiani:2004uc}. The \fzero\ is suggested to be a conventional $\rm{}q\bar{q}$~\cite{Chen:2003za}, compact tetraquark~\cite{Achasov:2020aun}, or $\rm{}K\overline{K}$ molecule~\cite{Ahmed:2020kmp}. In order to analyse the unknown structure of \fzero, several approaches are accessible in relativistic heavy-ion collisions in conjunction with the information in proton--proton (pp) collisions. 

The theory of strong interaction, quantum chromodynamics (QCD), predicts the formation of a state of strongly interacting matter, the so called quark-gluon plasma (QGP), in the hot and dense system reached in relativistic heavy-ion collisions. Many observations at the Large Hadron Collider (LHC), such as flow modulations~\cite{Bhalerao:2020ulk, ALICE:2019zfl}, jet quenching~\cite{ALICE:2019qyj}, and nuclear modification factors~\cite{ALICE:2019hno}, support the existence of the QGP~\cite{Adams:2005dq}. Specifically, the nuclear modification factors for different particle species in heavy-ion collisions show that the transverse momentum ($p_{\mathrm{T}}$) distribution of particles is strongly modified due to the presence of the hot and dense QGP medium. However, the nuclear modification factors are measured as close to unity in p--A collisions~\cite{ALICE:2016dei} for minimum-bias (MB) events, stating no substantial modification in p--A collisions in the high $p_{\mathrm{T}}$ range~($>$~8~GeV/$c$). 

The relative production yield of particles containing strange quarks is reported to increase faster than those with up and down quarks, which is usually referred to as ``strangeness enhancement''~\cite{ALICE:2016fzo} in high-multiplicity pp and p--Pb collisions. The observation of the strangeness enhancement, along with measurements of flow modulations in small collision systems~\cite{PHENIX:2018lia, ALICE:2021nir}, may hint at the formation of QGP droplets in small systems. The study of relative production yield for \fzero\ can be useful to check the strange quark content~\cite{2014ooi, 2014vbo} inside \fzero. On the other hand, the nuclear modification factors for baryons exhibit a strong enhancement at intermediate $p_{\mathrm{T}}$ (2~$<p_{\mathrm{T}}<$~5~GeV/$c$) compared with those for mesons which is mainly due to the number of constituent quarks (NCQ)~\cite{Cronin:1974zm, Fries:2003vb}. In addition, it has been suggested that the elliptic flows of different particles depend on the NCQ as demonstrated in Ref.~\cite{Wang:2022det}. These approaches can help to constrain the number of quarks forming \fzero.

The short-lived resonances, such as \rhoz~\cite{ALICE:2018qdv}, \kstar~\cite{ALICE:2019etb, ALICE:2016sak}, $\Sigma(1385)^{\pm}$~\cite{ALICE:2022zuc}, and $\Lambda$(1520)~\cite{ALICE:2018ewo} as well as \fzero, are good probes to study the properties of the system that results from the hadronisation of a QGP~\cite{Bierlich:2021poz}. In the late stage of the evolution of the system formed in heavy-ion collisions, there are two important temperatures and corresponding timescales: the chemical freeze-out, when the inelastic collisions among the constituents are expected to cease, and the later kinetic freeze-out, when all (elastic) interactions stop~\cite{Song:1996ik}. Because the duration between the chemical to the kinetic freeze-outs of the system is comparable with the lifetime of resonances~\cite{ALICE:2011dyt, ALICE:2019xyr}, the particles originating from those decays can actively interact with the hadronic gas via rescattering and regeneration processes, and such interactions result in modifications of resonance yields. The modifications can be observed by comparing the yield of resonances with those of long-lived or ground-state particles~\cite{ALICE:2018pal}. Measurements of $\rho/\pi$ and $\rm{}K^{*}/K$ yield ratios are good examples to study the properties of the late hadronic phase after the chemical freeze-out. Recently, system-size-dependent suppressions of particle yields are also observed in small systems~\cite{ALICE:2019etb}, suggesting rescattering and regeneration processes in small systems. The observation of the modification for the \fzero~yield may contribute to further understanding of the late hadronic phase.

%%--------------------

%The number of constituent quarks (NCQ)~\cite{Fries:2003vb} are supposed to affect to the flow modulations and the nuclear modification factor for different particle. The nuclear modification factors for baryons exhibits a strong enhancement at intermediate $p_{\mathrm{T}}$ (2~$<p_{\mathrm{T}}<$~5~GeV/$c$) compared with one for mesons which is mainly due to the NCQ~\cite{Cronin:1974zm}. 

%On the other hand, \fzero~production is also interesting due to its short lifetime~\cite{ParticleDataGroup:2020ssz} that can be also useful to study the hadronic final state and particle formation procedure.


%The elliptic flows scaled down by the NCQ are known to be comparable between different mesons and baryons~\cite{Wang:2022det}. Application of the NCQ scaling to \fzero\ is expected to provide insights into the nature of \fzero. The nuclear modification factors for baryons exhibits a strong enhancement at intermediate $p_{\mathrm{T}}$ (2~$<p_{\mathrm{T}}<$~5~GeV/$c$) compared with one for mesons which is mainly due to the NCQ~\cite{Cronin:1974zm}. Measurement of the nuclear modificaction factor for \fzero, therefore, can suggest the internal structure of \fzero.

 %One of models predicting the particle formation is the Canonical Statistical Model (CSM)~\cite{Vovchenko:2019kes}, which considers a system-size dependent chemistry at vanishing baryon density with conserved electric charge, baryon density, and strangeness locally, while allowing for under-saturation of strangeness.

In this paper, multiplicity-dependent measurement of anomalous suppression of \fzero~production in p--Pb collisions at $\snn$ = 5.02~TeV is reported for the first time. The measurement of \fzero~is conducted at midrapidity (-0.5~$<y<$~0) in 0~$<p_{\mathrm{T}}<$~8~GeV/$c$ for different multiplicity classes. In Sec.~\ref{sec:setup}, the experimental setup is described, while the reconstruction of \fzero\ and the relative corrections are explained in Sec.~\ref{sec:ana}. The study of systematic uncertainties for the measurement is reported in Sec.~\ref{sec:syst}. In Sec.~\ref{sec:results}, $p_{\mathrm{T}}$ spectra, particle yield ratios, and the nuclear modification factors are discussed. Finally, model comparisons and conclusions are described in Sec.~\ref{sec:summary}.

\label{sec:intro}



