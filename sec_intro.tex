% !TEX root = paper.tex

\section{Introduction}

Light scalar mesons, whose spin and parity are zero and even, respectively, are of particular interest as their nature can be explained with an exotic structure~\cite{ParticleDataGroup:2022pth}. Among them, a long-standing puzzle is related to the quark composition of the \fzero~particle~\cite{ExHIC:2010gcb, Jaffe:1976ig, Maiani:2004uc}. The \fzero~is suggested to be either a conventional meson ($\rm{}q\bar{q}$)~\cite{Chen:2003za}, a compact tetraquark~\cite{Achasov:2020aun}, or a $\rm{}K\overline{K}$ molecule~\cite{Ahmed:2020kmp}. By comparing different observables in heavy-ion collisions with those in pp interactions, the structure of the \fzero~can be probed.

The theory of the strong interaction, quantum chromodynamics (QCD), predicts the formation of a state of strongly interacting matter, the so-called quark--gluon plasma (QGP), under the conditions of high temperature and high energy density reached in relativistic heavy-ion collisions. Many observations at the Large Hadron Collider (LHC) and the Relativistic Heavy Ion Collider (RHIC), such as collective flow~\cite{Bhalerao:2020ulk, ALICE:2019zfl, Adams:2005dq, Adcox:2004mh} and jet quenching~\cite{ALICE:2019qyj, ATLAS:2010isq, PHENIX:2010nlr}, which is also manifest in the suppression of the yield of high-momentum hadrons~\cite{ALICE:2019hno, PHENIX:2006ujp} due to in-medium partonic energy loss, contribute to the understanding of the QGP properties~\cite{Heinz:2000bk, ALICE:2022wpn}. Specifically, the nuclear modification factors for different particle species, defined as the ratio of the transverse momentum ($p_{\mathrm{T}}$) distributions measured in heavy-ion collisions to the corresponding yields in pp interactions scaled by the number of nucleon--nucleon collisions, show a strong modification of the $p_{\mathrm{T}}$ spectra in large collision systems due to the presence of the hot and dense QGP medium. However, the nuclear modification factors are measured to be close to unity in minimum bias (MB) proton--nucleus (pA) collisions for $p_{\mathrm{T}}>$~8~GeV/$c$~\cite{ALICE:2016dei}, indicating no substantial modification in pA collisions in the high $p_{\mathrm{T}}$ range. 

Another effect observed in pp and pA collisions at the LHC is a multiplicity-dependent enhancement of the production of strange hadrons relative to hadrons composed of up and down quarks, which is usually referred to as ``strangeness enhancement''~\cite{ALICE:2016fzo}. The measurement of particle yield ratios of \fzero~to $\pi$ and \kstar~can be helpful to examine whether the \fzero~yield is influenced by the strangeness enhancement, thus providing sensitivity to the strange quark content inside the \fzero~\cite{LHCb:2014ooi, LHCb:2014vbo}. Moreover, other features observed in heavy-ion collisions, such as the strong enhancement of the nuclear modification factors of baryons at intermediate $p_{\mathrm{T}}$ (2~$<p_{\mathrm{T}}<$~5 GeV/$c$)~\cite{Fries:2003vb, ALICE:2022wpn} compared to those of mesons and the baryon and meson grouping of the elliptic flow~\cite{Wang:2022det}, show an apparent dependence on the number of constituent quarks (NCQ)~\cite{Wang:2022det}, reflecting the formation of hadrons from the QGP via quark coalescence~\cite{Fries:2003vb}. Hence, measurements of \fzero~production in systems where a QGP may be created can help to constrain the number of quarks forming the \fzero.

Short-lived resonances, such as \rhoz~\cite{ALICE:2018qdv}, \kstar~\cite{ALICE:2019etb, ALICE:2016sak}, $\Sigma(1385)^{\pm}$~\cite{ALICE:2022zuc}, and $\Lambda$(1520)~\cite{ALICE:2018ewo} as well as \fzero, are good probes to study the properties of the system that results from the hadronization of a QGP~\cite{Bierlich:2021poz, Knospe:2015nva}. In the late stage of the evolution of the system formed in heavy-ion collisions, there are two relevant temperatures and corresponding timescales: the chemical freeze-out, when the inelastic interactions among the constituents are expected to cease, and the later kinetic freeze-out, when all (elastic) interactions stop~\cite{Song:1996ik}. Since the time interval between the chemical and the kinetic freeze-outs of the system ($\sim$~10~fm/$c$) is comparable with the lifetime of resonances~\cite{ALICE:2011dyt, ALICE:2019xyr}, their decay products can actively interact with the hadronic gas via rescattering whereas regeneration can occur from interactions between particle pairs in the hadron gas. These two processes are designated as hadronic interactions in this Letter. The hadronic interactions result in modifications of resonance yields. The modifications can be studied by comparing the yield of resonances with those of long-lived or ground-state particles~\cite{ALICE:2018pal}. Measurements of \rhoz$/(\pi^{+}+\pi^{-})$~\cite{ALICE:2018qdv} and \kstar$\rm{}/(K^{+}+K^{-})$~\cite{ALICE:2019etb, ALICE:2016sak} yield ratios are good examples to study the properties of the late hadronic phase after the chemical freeze-out. It is worth mentioning that the ratios of particles with the same strangeness can eliminate potential strangeness enhancement effects in the ratio. Recently, system-size-dependent modifications of particle yields are also observed in small collision systems~\cite{ALICE:2016sak, ALICE:2019etb}, suggesting that rescattering and regeneration may also occur in high-multiplicity pp and p--Pb collisions. These hadronic interactions depend on the hadronic cross section of the decay products inside the hadronic medium, the lifetimes of the resonance, and the duration of the hadronic phase. The suppression of resonance yields in the hadronic gas can be explained by rescattering dominating over regeneration. In addition, the final states of resonances decaying to $\pi\pi$, such as \fzero, \rhoz, and f$_{2}$(1270), are affected by the same cross section of pions and the medium, while the amount of hadronic interactions differs due to different lifetimes of these resonances. In this context, measuring the modification of the \fzero~yield may contribute to further understanding of the late hadronic phase.

In this Letter, multiplicity-dependent measurements of \fzero~production in p--Pb collisions at center-of-mass energy per nucleon--nucleon collision $\snn$ = 5.02~TeV are reported for the first time. The \fzero~is measured at midrapidity ($-0.5<y<0$) in 0~$<p_{\mathrm{T}}<$~8~GeV/$c$ for different multiplicity classes. In Sec.~\ref{sec:setup}, the experimental setup is described, while the reconstruction of \fzero\ and the relative corrections are explained in Sec.~\ref{sec:ana}. The study of systematic uncertainties for the measurement is reported in Sec.~\ref{sec:syst}. In Sec.~\ref{sec:results}, $p_{\mathrm{T}}$ spectra, particle yield ratios, the nuclear modification factors, and model comparisons are discussed. Finally, conclusions are outlined in Sec.~\ref{sec:summary}.

\label{sec:intro}



