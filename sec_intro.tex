% !TEX root = paper.tex

\section{Introduction}

The theory of the strong interaction, Quantum chromodynamics (QCD) predicts a strongly interacting matter that is so-called quark-gluon plasma (QGP) in extremely high temperature and energy-density conditions. Such conditions have reached a climax  with ultra-relativistic heavy-ion collisions at the Large Hadron Collider (LHC)~\cite{Wolschin:2020qxa}. Many observations at the LHC such as flow modulations~\cite{Bhalerao:2020ulk, ALICE:2019zfl}, jet quenching~\cite{ALICE:2019qyj}, and nuclear modification factors~\cite{ALICE:2019hno} support the existence of the QGP~\cite{Adams:2005dq} repeatably, so far. 
%Specifically, the nuclear modification factor is one of the strongest tool to study the properties of the QGP.
Specifically, the nuclear modification factors for different particles in heavy-ion collisions show that the behaviour of particles is strongly modified in the hot and dense medium while those in p--A collisions~\cite{ALICE:2016dei} are measured as unity in minimum-bias events stating no strong modification in p--A collisions in a  high transverse momentum range~($p_{\rm{T}}>$~3~GeV/$c$). 

On the other hand, the short-lived resonances such as $\rm{}K^{*}$(892)~\cite{ALICE:2019etb, ALICE:2016sak}, $\rho$(770)~\cite{ALICE:2018qdv}, and $\Lambda$(1520)~\cite{ALICE:2018ewo} are good probes to study the properties of the dynamic evolution of the QGP in its late stage~\cite{Bierlich:2021poz}. The evolution time from the chemical to kinetic freeze-outs is compatible with the lifetime of resonances~\cite{ALICE:2011dyt, ALICE:2019xyr}. The daughter particles from them can actively interact with the hadronic gas, resulting in modifications of resonance yields. Such modifications can be observed by comparing the yield of resonances with those of long-lived or ground-state particles~\cite{ALICE:2018pal}. Measurements of $\rho/\pi$ and $\rm{}K^{*}/K$ yield ratios are good examples to study the properties of the late hadronic phase after the chemical freeze-out.

Light scalar mesons, whose spin and parity are zero and even respectively, are of particular interest as their natures can be explained with an exotic structure~\cite{ParticleDataGroup:2020ssz}. Among those, the understanding of \fzero\ particles lies in a long-standing puzzle in terms of its quark content, where many discussions can be found in Refs.~\cite{ExHIC:2010gcb, Jaffe:1976ig, Maiani:2004uc}. The structure of \fzero\ is suggested to be a conventional $\rm{}q\bar{q}$~\cite{Chen:2003za}, compact tetraquark~\cite{Achasov:2020aun}, or $\rm{}K\overline{K}$ molecule~\cite{Ahmed:2020kmp}. In order to analyse the unknown structure of \fzero, several methods can be employed. 

The relative production rates of particles containing strange quark is reported to increase faster than those with up and down quarks that is usually referred to as ``strangeness enhancement''~\cite{ALICE:2016fzo}. The study of relative production rate for \fzero\ can be useful to check the strange quark content inside \fzero. On the other hand, the nuclear modification factors for baryons exhibit a strong enhancement at intermediate $p_{\mathrm{T}}$ (2~$<p_{\mathrm{T}}<$~5~GeV/$c$) compared with one for mesons which is mainly due to the number of constituent quarks (NCQ)~\cite{Cronin:1974zm,Fries:2003vb}. This approach can help us to constrain the number of quarks forming \fzero. However, these studies need to be approached carefully as the \fzero\ production is also affected during the hadronic final state and particle formation procedure due to its short lifetime~\cite{ParticleDataGroup:2020ssz}.  

%The number of constituent quarks (NCQ)~\cite{Fries:2003vb} are supposed to affect to the flow modulations and the nuclear modification factor for different particle. The nuclear modification factors for baryons exhibits a strong enhancement at intermediate $p_{\mathrm{T}}$ (2~$<p_{\mathrm{T}}<$~5~GeV/$c$) compared with one for mesons which is mainly due to the NCQ~\cite{Cronin:1974zm}. 

%On the other hand, \fzero~production is also interesting due to its short lifetime~\cite{ParticleDataGroup:2020ssz} that can be also useful to study the hadronic final state and particle formation procedure.


%The elliptic flows scaled down by the NCQ are known to be comparable between different mesons and baryons~\cite{Wang:2022det}. Application of the NCQ scaling to \fzero\ is expected to provide insights into the nature of \fzero. The nuclear modification factors for baryons exhibits a strong enhancement at intermediate $p_{\mathrm{T}}$ (2~$<p_{\mathrm{T}}<$~5~GeV/$c$) compared with one for mesons which is mainly due to the NCQ~\cite{Cronin:1974zm}. Measurement of the nuclear modificaction factor for \fzero, therefore, can suggest the internal structure of \fzero.

 %One of models predicting the particle formation is the Canonical Statistical Model (CSM)~\cite{Vovchenko:2019kes}, which considers a system-size dependent chemistry at vanishing baryon density with conserved electric charge, baryon density, and strangeness locally, while allowing for under-saturation of strangeness.

In this paper, multiplicity dependent measurement of abnormal suppression of \fzero~production in p--Pb collisions at $\snn$ = 5.02~TeV is reported. The measurement of \fzero~is conducted at midrapidity (-0.5~$<\mathrm{y}<$~0) in a transverse momentum ($p_{\mathrm{T}}$) range of 0~$<p_{\mathrm{T}}<$~8~GeV/$c$ for different multiplicity classes. In Sec.~\ref{sec:setup}, ALICE apparatus is described, and the reconstruction of \fzero~and correction is explained in Sec.~\ref{sec:ana}. Systematic study for the measurement is reported in Sec.~\ref{sec:syst}. In Sec.~\ref{sec:results}, $p_{\mathrm{T}}$ spectra, particle yield ratios, and the nuclear modification factors are discussed. Finally, model comparisons and conclusions are described in Sec.~\ref{sec:summary}.

\label{sec:intro}



