% !TEX root = paper.tex

\section{Introduction}

Light scalar mesons, whose spin and parity are zero and even, respectively, are of particular interest as their nature can be explained with an exotic structure~\cite{ParticleDataGroup:2022pth}. Among them, the understanding of \fzero\ particle is a long-standing puzzle in terms of its quark content~\cite{ExHIC:2010gcb, Jaffe:1976ig, Maiani:2004uc}. The \fzero\ is suggested to be either a conventional meson ($\rm{}q\bar{q}$)~\cite{Chen:2003za}, a compact tetraquark~\cite{Achasov:2020aun}, or a $\rm{}K\overline{K}$ molecule~\cite{Ahmed:2020kmp}. In order to analyze the unknown structure of \fzero, several approaches are accessible in relativistic heavy-ion collisions in conjunction with the corresponding information from proton--proton (pp) collisions. 

The theory of the strong interaction, quantum chromodynamics (QCD), predicts the formation of a state of strongly interacting matter, the so-called quark--gluon plasma (QGP), under the conditions of high temperature and high energy density reached in relativistic heavy-ion collisions. Many observations at the Large Hadron Collider (LHC) and the Relativistic Heavy Ion Collider (RHIC), such as collective flow~\cite{Bhalerao:2020ulk, ALICE:2019zfl, Adams:2005dq, Adcox:2004mh} and partonic in-medium energy loss indicated by jet quenching~\cite{ALICE:2019qyj, ATLAS:2010isq, PHENIX:2010nlr} and the suppression of energetic hadron yields~\cite{ALICE:2019hno, PHENIX:2006ujp}, contribute to the understanding of the QGP properties~\cite{Heinz:2000bk, ALICE:2022wpn}. Specifically, the nuclear modification factors for different particle species, defined as the ratio of the transverse momentum ($p_{\mathrm{T}}$) distributions measured in heavy-ion collisions to the corresponding yields in pp interactions scaled by the number of nucleon--nucleon collisions, show a strong modification of the $p_{\mathrm{T}}$ spectra in Pb--Pb collisions due to the presence of the hot and dense QGP medium. However, the nuclear modification factors are measured to be close to unity in minimum bias (MB) in proton--nucleus (pA) collisions~\cite{ALICE:2016dei}, stating no substantial modification in pA collisions in the high $p_{\mathrm{T}}$ range. 

The multiplicity-dependent enhanced production of strange hadrons relative to hadrons composed of up and down quarks is usually referred to as ``strangeness enhancement''~\cite{ALICE:2016fzo}. The observation of the strangeness enhancement, along with measurements of flow modulations in pp, p--Pb, and d--Au collisions~\cite{PHENIX:2018lia, ALICE:2021nir, ATLAS:2016yzd}, may hint at the formation of QGP droplets in small collision systems. In this context, the measurement of particle yield ratios of \fzero~to $\pi$ and \kstar~can be helpful to check the strange quark content~\cite{LHCb:2014ooi, LHCb:2014vbo} inside \fzero. Moreover, other features observed in heavy-ion collisions, such as the strong enhancement of the nuclear modification factors of baryons at intermediate $p_{\mathrm{T}}$ (2~$<p_{\mathrm{T}}<$~5 GeV/$c$)~\cite{Fries:2003vb, ALICE:2022wpn} compared to those of mesons and the baryon and meson grouping of the elliptic flow~\cite{Wang:2022det}, show an apparent dependence on the number of constituent quarks (NCQ)~\cite{Wang:2022det}, reflecting the formation of hadrons from the QGP via quark coalescence~\cite{Fries:2003vb}. Hence, measurements of \fzero~production in systems where a QGP may be created can help to constrain the number of quarks forming~\fzero.

Short-lived resonances, such as \rhoz~\cite{ALICE:2018qdv}, \kstar~\cite{ALICE:2019etb, ALICE:2016sak}, $\Sigma(1385)^{\pm}$~\cite{ALICE:2022zuc}, and $\Lambda$(1520)~\cite{ALICE:2018ewo} as well as \fzero, are good probes to study the properties of the system that results from the hadronization of a QGP~\cite{Bierlich:2021poz, Knospe:2015nva}. In the late stage of the evolution of the system formed in heavy-ion collisions, there are two important temperatures and corresponding timescales: the chemical freeze-out, when the inelastic collisions among the constituents are expected to cease, and the later kinetic freeze-out, when all (elastic) interactions stop~\cite{Song:1996ik}. Since the time interval between the chemical and the kinetic freeze-outs of the system ($\sim$~10~fm/$c$) is comparable with the lifetime of resonances~\cite{ALICE:2011dyt, ALICE:2019xyr}, their decay products can actively interact with the hadronic gas via rescattering and regeneration processes, which are designated as hadronic interactions in this Letter. The hadronic interactions result in modifications of resonance yields. The modifications can be studied by comparing the yield of resonances with those of long-lived or ground-state particles~\cite{ALICE:2018pal}. Measurements of \rhoz$/(\pi^{+}+\pi^{-})$~\cite{ALICE:2018qdv} and \kstar$\rm{}/(K^{+}+K^{-})$~\cite{ALICE:2019etb, ALICE:2016sak} yield ratios are good examples to study the properties of the late hadronic phase after the chemical freeze-out. Recently, system-size-dependent modifications of particle yields are also observed in small systems~\cite{ALICE:2016sak, ALICE:2019etb}, suggesting that rescattering and regeneration may also occur in high-multiplicity pp or p--Pb collisions. The suppression of resonance yields modified in the hadronic gas can be explained by rescattering dominating over regeneration. Observing the modification for the \fzero~yield may contribute to further understanding the late hadronic phase.

In this Letter, multiplicity-dependent measurement of \fzero~production in p--Pb collisions at center-of-mass energy per nucleon--nucleon collision $\snn$ = 5.02~TeV is reported for the first time. The measurement of \fzero~is conducted at midrapidity (-0.5~$<y<$~0) in 0~$<p_{\mathrm{T}}<$~8~GeV/$c$ for different multiplicity classes. In Sec.~\ref{sec:setup}, the experimental setup is described, while the reconstruction of \fzero\ and the relative corrections are explained in Sec.~\ref{sec:ana}. The study of systematic uncertainties for the measurement is reported in Sec.~\ref{sec:syst}. In Sec.~\ref{sec:results}, $p_{\mathrm{T}}$ spectra, particle yield ratios, the nuclear modification factors, and model comparisons are discussed. Finally, conclusions are described in Sec.~\ref{sec:summary}.


\label{sec:intro}



