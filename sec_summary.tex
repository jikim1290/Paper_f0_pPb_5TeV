% !TEX root = paper.tex

\section{Conclusions}
\label{sec:summary}

The multiplicity dependent \fzero~production in p--Pb collisions at \snn~=~5.02~TeV is presented. \fzero~is reconstructed via \fzero~$\rightarrow\pi^{+}\pi^{-}$ decay channel in midrapidity (-0.5~$<\mathrm{y}<$~0) for the 0~$<p_{\mathrm{T}}<$~8~GeV/$c$ range. Measure $p_{\mathrm{T}}$ spectra show increasing trend of the yield and mean $p_{\mathrm{T}}$. The particle yield ratio of \fzero~to $\pi$ is decreasing with the system size and the suppression exists only at low $p_{\mathrm{T}}<$~4~GeV/$c$, which can be explained by rescattering effects. The CSM overestimates the \fzero/$\pi$ and does not describe the decreasing trend because the CSM does not consider rescattering effects. The particle yield ratio of \fzero~to $K^{*0}$ is also decreasing with the system size but the suppression exists at the entire $p_{\mathrm{T}}$ range, which is disparate dependency from rescattering effects. The CSM qualitatively estimates the decreasing trend for \fzero/$K^{*0}$ with the assumption of no hidden strangeness for \fzero, while overestimating the \fzero/$K^{*0}$ with the assumption of two hidden strangeness. The result implies that $K^{*0}$ is relatively enhanced compared with \fzero due to the strangeness enhancement. Furthermore, the multiplicity dependent nuclear modification factor for \fzero~exhibits strong suppression at low $p_{\mathrm{T}}$, and the suppression is dependent on the multiplicity class, which is the observation of rescattering effects. Additionally, no Cronin peak is observed from the nuclear modification factor even in high-multiplicity events. This suggests the internal structure of \fzero~to be $q\bar{q}$.