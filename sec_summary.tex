% !TEX root = paper.tex

\section{Conclusions}
\label{sec:summary}

The multiplicity and $p_{\mathrm{T}}$ dependence of \fzero~production in p--Pb collisions at \snn~=~5.02~TeV is presented. The \fzero~is reconstructed via the \fzero~$\rightarrow\pi^{+}\pi^{-}$ decay channel at midrapidity ($-0.5<y<0$) in the transverse momentum region of $0<p_{\mathrm{T}}<8$~GeV/$c$. A hardening of the $p_{\mathrm{T}}$ spectra and a consequent increase of the mean $p_{\mathrm{T}}$ are observed with increasing multiplicity. 

The $p_{\mathrm{T}}$-integrated particle yield ratio of \fzero~to $\pi$ is decreasing with increasing multiplicities, and the $p_{\mathrm{T}}$-differential studies show a clear suppression of the \fzero~to $\pi$ ratio for $p_{\mathrm{T}}<$~3.5~GeV/$c$, indicating that rescattering effects for \fzero~particles exist in p--Pb collisions. The CSM overestimates the \fzero/$\pi$ ratio, and it does not describe the decreasing trend because the CSM does not consider rescattering processes. The $p_{\mathrm{T}}$-integrated \fzero/\kstar~yield ratio also decreases with increasing multiplicity. The suppression of the \fzero~to \kstar~ratio is observed in the entire measured $p_{\mathrm{T}}$ range, showing a different $p_{\mathrm{T}}$ dependency relative to the one expected from a rescattering scenario. The CSM qualitatively describes the decreasing trend for the $p_{\mathrm{T}}$-integrated \fzero/\kstar~ratio as a function of multiplicity with the assumption of no hidden strangeness for \fzero, while it overestimates the \fzero/\kstar~with the assumption of two strange quarks. These results indicate that the production of \kstar~is relatively enhanced compared with \fzero~due to the strangeness enhancement. 


Additionally, $p_{\mathrm{T}}$-differential \fzero/\kstar~ratio does not exhibit the enhancement of baryon-to-meson ratio, suggesting the number of constituents quarks for \fzero~to be two. Furthermore, the multiplicity-dependent nuclear modification factor ($Q_{\mbox{pPb}}$) for \fzero~exhibits a strong suppression at low $p_{\mathrm{T}}$, and the suppression depends on the multiplicity class, which can be explained by the rescattering effects. In addition, no Cronin-like enhancement is observed in $Q_{\mbox{pPb}}$, even in high-multiplicity events. The absence of the Cronin-like enhancement suggests that the number of constituent quarks of the \fzero~particle is two. 

The abnormal suppression in terms of multiplicity and transverse momentum relative to other particles sheds light on the internal structure of \fzero~to be a conventional meson with no hidden strange quarks and provides insight into the properties of the late hadronic phase in p--Pb collisions.
