% !TEX root = paper.tex

\section{Conclusions}
\label{sec:summary}

The multiplicity-dependent \fzero~production in p--Pb collisions at \snn~=~5.02~TeV is presented. The \fzero~is reconstructed via the \fzero~$\rightarrow\pi^{+}\pi^{-}$ decay channel in midrapidity ($-0.5<y<0$) in a transverse momentum region of $0<p_{\mathrm{T}}<8$~GeV/$c$. The $p_{\mathrm{T}}$ spectra of \fzero~particles show an increasing trend of the yield and the mean $p_{\mathrm{T}}$ for higher multiplicity events. The particle yield ratio of \fzero~to $\pi$ is decreasing with increasing multiplicities, and the suppression of the \fzero~to $\pi$ ratio is observed only at low $p_{\mathrm{T}}<$~4~GeV/$c$, indicating that the rescattering effects exist for \fzero~particles. The Canonical Statistical Model (CSM) overestimates the \fzero/$\pi$, and it does not describe the decreasing trend because the CSM does not consider rescattering effects. The particle yield ratio of \fzero~to $\rm{}K^{*0}$ also decreases with increasing multiplicities. The suppression of the \fzero~to $\rm{}K^{*0}$ ratio is observed in the entire $p_{\mathrm{T}}$ region, showing a different $p_{\mathrm{T}}$ dependency relative to the one expected from a rescattering scenario. The CSM qualitatively estimates the decreasing trend for \fzero/$\rm{}K^{*0}$ with the assumption of no hidden strangeness for \fzero, while it overestimates the \fzero/$\rm{}K^{*0}$ with the assumption of the two hidden strangeness. The result implies that $\rm{}K^{*0}$ is relatively enhanced compared with \fzero~due to the strangeness enhancement. Furthermore, the multiplicity-dependent nuclear modification factor for \fzero~exhibits a strong suppression at a low $p_{\mathrm{T}}$, and the suppression depends on the multiplicity class, which can be explained by the rescattering effects. Additionally, no Cronin-like enhancement is observed from the nuclear modification factor, even in high-multiplicity events. This suggests that the number of constituent quarks of the \fzero~particle is smaller than three.